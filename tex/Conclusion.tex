\section{Conclusion}\label{sec:Conclusion}

In this project, we have implemented data types for regular expressions, deterministic automata, and non-deterministic automata. 
Using these, we implemented the constructions that are used to prove important results in automata theory: 
the expressive equivalence between regular languages, NFAs, and DFAs.

Haskell's type system and the \texttt{Maybe} monad allowed us to effectively model partial transition functions for DFAs by using \texttt{Maybe state} to map arguments that have no value specified, 
meaning no transition for some symbols, to a dummy \enquote{garbage} state. 
The only downside is that the models are a bit more complicated to write, and the code for \texttt{evaluateDFA} is slightly more complicated,
which we think is an acceptable trade-off.
Here, one could add a a translation function that transforms a given input list to a function of the right type. 
As for NFAs, using the \texttt{Maybe} type for symbols allowed us to separately treat the $\varepsilon$-transitions for NFAs 
by singling out an object in a non-specified type to work as $\varepsilon$. 

Currently, the NFA to DFA translation is split into two parts: 
first, we translate the DFA using the powerset construction, and then minimize it. 
By producing the minimized DFA on-the-fly and generating only the states we can transition to without computing the entire power set, 
we could improve the efficiency of the translation.

As for translating from regular expressions to NFAs, this was straightforward: \enquote{glue} together the recursively generated NFAs
while keeping track of fresh state labels to use for new states, using \texttt{Int} as our state type.
In contrast, the other direction was much trickier, requiring the constraint of \texttt{Int}-labeled states for the helper function \texttt{r k i j} 
defining all \texttt{k}-bounded paths from \texttt{i} to \texttt{j}. 
Thus, our translations between regular expressions and NFAs are both limited to the case where states are labled with \texttt{Int}.
We believe this is again an acceptable trade-off to have simpler code, as the type used for states only acts as a labelling scheme for states in our functions.
If however in the future we wished to extend this to NFAs with other state labels,
we could write simple coercions between arbitrary and \texttt{Int} labels and work \enquote{under the hood} solely with the \texttt{Int} labels.

Our test suite shows that the algorithms work in many cases, 
but due to the inefficiency of some of the algorithms, such as the translation \texttt{nfaToReg}, 
we cannot test for long expressions. 




% Finally, we can see that \cite{liuWang2013:agentTypesHLPE} is a nice paper.
