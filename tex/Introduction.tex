For this project, we have set three main goals: 
first, implement data types for deterministic and non-deterministic finite automata, as well as for regular expressions,
while mainting as much generality as possible;
second, implement conversions between them that preserve the language that they describe;
third, test our construction using arbitrarily generated inputs.
Regular languages are defined as those accepted by a deterministic finite automaton
- if we look at the bigger picture, then, our goal is to implement the constructions
that are used to prove that DFAs, NFAs and regular expressions have the same expressive power
in describing regular languages. 
More specifically, we mostly follow Chapter 1 of~\cite{sipser2012}.

Our report is structured as follows.
In Section~\ref{sec:DfaAndNfa}, we provide an implementation of data types for DFAs and NFAs
and we implement the powerset construction, 
which can be used to prove that DFAs have the same expressive power as NFAs.
Next, in Section~\ref{sec:RegExp}, we implement the data type for regular expressions.
In Section~\ref{sec:equivalence_regex_fa}, we implement the conversions from regular expressions to NFAs
and from NFAs to regular expressions,
which are used to show that regular expressions also describe regular languages.
In Section~\ref{sec:tests}, we describe our test suite.
Finally, we provide a small demo in Section~\ref{sec:executable},
and draw some conclusions in Section~\ref{sec:Conclusion}.