\section{Equivalence of finite automata and regular expressions}\label{sec:equivalence_regex_fa}

In this section, our goal is to implement the constructive proof of Theorem 1.54 from~\cite{sipser2012}.

\begin{theorem}
    A language is regular if and only if it is described by a regular expression.
\end{theorem}

Using the fact that DFAs have the same expressive power as NFAs, we will implement conversions
from regular expressions to NFAs and back.
In particular, in \S~\ref{subsec:RegToNfa}, we implement the construction of an NFA from a regular expression 
which can be used to formally prove that if a language is described by a regular expression, then it is regular.
Next, in \S~\ref{subsec:NfaToReg}, we implement the construction of a regular expression from a given NFA
via Kleene's algorithm,
which shows that if a language is regular, then it is described by a regular expression.