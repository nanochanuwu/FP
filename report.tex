\documentclass[12pt,a4paper]{article}
\usepackage{etex,datetime,setspace,latexsym}
\usepackage{amssymb,amsmath,amsthm}
\newtheorem{theorem}{Theorem}
\newtheorem{lemma}[theorem]{Lemma}
\theoremstyle{definition}
\newtheorem{definition}[theorem]{Definition}
\usepackage{fancybox,dialogue,float,wrapfig,enumerate,microtype}
\usepackage{verbatim,xcolor,multicol,titlesec,tabularx,mdframed}

\usepackage[utf8]{inputenc}
\usepackage[pdftex]{hyperref}
\usepackage[margin=2cm,bottom=3cm,footskip=15mm]{geometry}
\parindent0cm
\parskip1em

\usepackage{tikz}
\usetikzlibrary{arrows,trees,positioning,shapes,patterns}
\usetikzlibrary{intersections,calc,fpu,decorations.pathreplacing}

\usepackage[T1]{fontenc} % better fonts

% Haskell code listings in our own style
\usepackage{listings,color}
\definecolor{lightgrey}{gray}{0.35}
\definecolor{darkgrey}{gray}{0.20}
\definecolor{lightestyellow}{rgb}{1,1,0.92}
\definecolor{dkgreen}{rgb}{0,.2,0}
\definecolor{dkblue}{rgb}{0,0,.2}
\definecolor{dkyellow}{cmyk}{0,0,.7,.5}
\definecolor{lightgrey}{gray}{0.4}
\definecolor{gray}{gray}{0.50}
\lstset{
  language        = Haskell,
  basicstyle      = \scriptsize\ttfamily,
  keywordstyle    = \color{dkblue},     stringstyle     = \color{red},
  identifierstyle = \color{dkgreen},    commentstyle    = \color{gray},
  showspaces      = false,              showstringspaces= false,
  rulecolor       = \color{gray},       showtabs        = false,
  tabsize         = 8,                  breaklines      = true,
  xleftmargin     = 8pt,                xrightmargin    = 8pt,
  frame           = single,             stepnumber      = 1,
  aboveskip       = 10pt plus 3pt,
  belowskip       = 1pt plus 1pt
}
\lstnewenvironment{code}[0]{}{}
\lstnewenvironment{showCode}[0]{\lstset{numbers=none}}{} % only shown, not compiled

% will the real phi please stand up
\renewcommand{\phi}{\varphi}

% load hyperref as late as possible for compatibility
\usepackage[pdftex]{hyperref}
\hypersetup{
  pdfborder = {0 0 0},
  breaklinks = true,
  linktoc = all,
}
\pdfinfoomitdate=1
\pdftrailerid{}
\pdfsuppressptexinfo15

\usepackage{csquotes}

\title{Finite automata and regular expressions in Haskell}
\author{N. Batistoni, M. L\"ursen, P. Fink, J. Sorkin}
\date{\today}
\hypersetup{pdfauthor={Me}, pdftitle={My Report}}

\begin{document}

\maketitle

\begin{abstract}
    We implement the data types for deterministic and non-deterministic finite automata,
    as well as for regular expressions.
    Moreover, we implement the proofs of their equivalence in describing regular languages
    from Chapter 1 of~\cite{sipser2012}.

    Notes for the peer reviewers: \begin{itemize}
        \item to compile everything, use \verb|stack build|;
        \item to play with the code in ghci, use \verb|stack ghci|;
        \item there is an executable, but it is still the one from the template which doesn't really do anything;
        \item to run the tests from \S~\ref{sec:tests}, use \verb|stack clean && stack test|. 
    \end{itemize}
\end{abstract}

\vfill

\tableofcontents

\clearpage

% We include one file for each section. The ones containing code should
% be called something.lhs and also mentioned in the .cabal file.

% 
\section{How to use this?}

To generate the PDF, open \texttt{report.tex} in your favorite \LaTeX editor and compile.
Alternatively, you can manually do
\texttt{pdflatex report; bibtex report; pdflatex report; pdflatex report} in a terminal.

You should have stack installed (see \url{https://haskellstack.org/}) and
open a terminal in the same folder.

\begin{itemize}
  \item To compile everything: \verb|stack build|.
  \item To open ghci and play with your code: \verb|stack ghci|
  \item To run the executable from Section \ref{sec:Main}: \verb|stack build && stack exec myprogram|
  \item To run the tests from Section \ref{sec:simpletests}: \verb|stack clean && stack test --coverage|
\end{itemize}


\input{lib/DfaAndNfa.lhs}

\input{lib/NfaToDfa.lhs}

\input{lib/RegExp.lhs}

\section{Equivalence of finite automata and regular expressions}\label{sec:equivalence_regex_fa}

In this section, our goal is to implement the constructive proof of Theorem 1.54 from~\cite{sipser2012}.

\begin{theorem}
    A language is regular if and only if it is described by a regular expression.
\end{theorem}

In \S~\ref{subsec:RegToNfa}, we implement the construction of an NFA from a regular expression 
which shows that if a language is described by a regular expression, then it is regular.
Next, in \S~\ref{subsec:NfaToReg}, we implement the construction of a regular expression from a given NFA
that shows that if a language is regular, then it is described by a regular expression.

\input{lib/RegToNfa.lhs}

\input{lib/NfaToReg.lhs}

% \input{exec/Main.lhs}

\input{test/tests.lhs}

\section{Conclusion}\label{sec:Conclusion}

In this project, we have implemented data types for regular expressions, deterministic automata, and non-deterministic automata. 
Using these, we demonstrated important results in automata theory: the expressive equivalence between regular languages, NFAs, and DFAs.

Haskell's type system and the Maybe monad allowed us to effectively model partial functions by using Maybe states to map arguments that have no value specified, meaning no transition for some symbols, to a dummy state. The only downside is that the models are a bit more complicated to write because one has to use the \texttt{maybeMap}. Here, one could add a a translation function that transforms a given input list to function of the right type. 
Using the Maybe type for symbols allowed us to separately treat the $\epsilon$-transitions for NFAs by singling out an object in a non-specified type to work as $\epsilon$. 

Currently, the NFA to DFA translation is split into two parts: first, we translate the DFA using the powerset construction and then minimize it. By producing the minimized DFA on-the-fly and generating only the states we can transition to without computing the entire power set, we could the efficiency of the translation.

Something on REGEX

Tests show that the algorithms work in many cases, but due to the inefficiency of some of the algorithms, such as the translation from NfaToReg, we cannot test for long expressions. 




% Finally, we can see that \cite{liuWang2013:agentTypesHLPE} is a nice paper.


\addcontentsline{toc}{section}{Bibliography}
\bibliographystyle{alpha}
\bibliography{references.bib}

\end{document}
